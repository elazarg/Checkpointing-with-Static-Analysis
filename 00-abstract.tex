\begin{abstract}
We present a static analysis framework for Python programs that enables automatic synthesis of selective program checkpointing instrumentation, achieving checkpoint size reduction of 1--3 orders of magnitude compared to system-level alternatives (VM or process). The main challenge is Python's dynamic nature; our approach employs abstract interpretation for type inference combined with points-to analysis, liveness analysis, and lightweight effect annotations for both built-in functions and external libraries. This enables cohesive collaboration between analyses, with mutual exchange of information improving both precision and robustness. To demonstrate our contribution, we present a specialization that combines standard analyses to automatically identify the minimal set of local variables needed for crash-recovery in iterative algorithms.

We evaluate the framework's effectiveness on realistic numerical Python programs: K-Means clustering, Orthogonal Matching Pursuit, and clique enumeration. Our results demonstrate that heap analysis with library-agnostic effect annotations can successfully handle real-world Python despite its dynamic features. When carefully applied to idiomatic, structurally constrained programs, these techniques enable significant automated optimizations for an important class of numerical computing applications.
\end{abstract}