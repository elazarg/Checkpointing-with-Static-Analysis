\begin{abstract}
Static analysis of Python programs faces significant challenges due to dynamic typing, reflection, and flexible control structures. However, important classes of Python programs --- particularly in numerical computing --- often avoid these problematic features, using explicit type annotations and predictable control flow patterns that may enable effective static reasoning.

We demonstrate that standard program analysis techniques can be successfully applied to well-structured Python programs to enable practical optimizations. Our approach combines points-to analysis, type inference, and liveness analysis, augmented with lightweight effect annotations for built-in types and external libraries. To demonstrate the framework's utility, we develop a specialization that performs escape analysis across loop iteration boundaries, automatically identifying minimal persistent state for iterative algorithms.

We apply this analysis to synthesize selective checkpointing instrumentation for realistic numerical Python programs, including k-means clustering, orthogonal matching pursuit, and graph search algorithms. Our evaluation shows that this approach can achieve checkpoint size reductions of 2--5 orders of magnitude compared to system-level alternatives, demonstrating that established static analysis techniques --- when carefully applied to structurally constrained Python code --- can enable significant automated optimizations.
\end{abstract}
