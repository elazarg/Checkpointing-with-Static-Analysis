\begin{abstract}
We present a static analysis framework for Python programs that enables automatic synthesis of selective program checkpointing instrumentation, achieving checkpoint size reduction of 1--3 orders of magnitude compared to system-level alternatives (VM or process). The main challenge is the dynamic nature of Python; our approach employes Abstract Interpretation to provide sound type inference that can be combines with other analyses, such as points-to analysis and liveness analysis Lightweight effect annotations for built-in function and external libraries. Notably, this provides a cohesive collaboration between type inference and the other analyses, and mutual exchange of information to make both more precise and robust. To demonstrate our contribution, we present a specialization that showcases a collaboration of standard analyses to automatically identify the smallest set of local variables needed for crash-recovery in iterative algorithms.

We evaluate the framework's effectiveness on realistic numerical Python programs: K-Means clustering, orthogonal matching pursuit, and clique enumeration. While static analysis of Python traditionally faces challenges due to its dynamic features, we show that established static analysis techniques, when carefully applied to idiomatic, structurally constrained Python code, can enable significant automated optimizations for an important class of numerical computing applications.
\end{abstract}
