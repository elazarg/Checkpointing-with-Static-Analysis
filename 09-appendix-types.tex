\section{Type System}
\label{appendix:typesystem}
\section*{Appendix: Type System Formalization}
This appendix provides a formal specification of the complete type system used in our analysis framework, building on the simpler overview demonstrated in the main paper (Section X). When combined with the pointer analysis detailed in Appendix B, this system enables precise modeling of dynamic behaviors in scientific Python code, such as array mutations in NumPy or side effects in SciPy simulations. The full implementation is available in \texttt{type_system.py}.

\subsection{Syntax}

\paragraph{Types.} The set of type expressions $\mathcal{T}$ is defined by:
\[
\begin{aligned}
\tau ::= \;& c \mid X \mid X^* \mid \ell \mid \tau_1 + \tau_2 \\
&\mid R \mid f : \tau \\
&\mid \mathsf{class}(C, R, \overline{\tau}, \overline{X}, \mathsf{protocol}?) \\
&\mid \forall \overline{X}.R \xrightarrow{\epsilon} \tau \mid \{\varphi_1, \ldots, \varphi_n\} \\
&\mid \tau\langle\overline{\tau}\rangle \mid \tau.\ell \\
R ::= \;& \{f_1 : \tau_1, \ldots, f_n : \tau_n\}
\end{aligned}
\]

where:
\begin{itemize}
\item $c$ ranges over type constants (qualified names like \texttt{builtins.int})
\item $X, X^*$ are type variables (ordinary and variadic)  
\item $\ell$ are literal types with values from $\mathbb{Z} \cup \mathbb{S} \cup \mathbb{B} \cup \{\mathsf{None}\} \cup \ldots$
\item $\tau_1 + \tau_2$ is union type
\item $R$ is record type with field entries $f_i$
\item $f : \tau$ is a field entry (row)
\item $\mathsf{class}(C, R, \overline{\tau}, \overline{X}, \mathsf{protocol}?)$ defines class $C$ with member record $R$, supertypes $\overline{\tau}$, type parameters $\overline{X}$, and optional protocol flag
\item $\forall \overline{X}.R \xrightarrow{\epsilon} \tau$ is polymorphic function type with parameter record $R$, effect $\epsilon$, and return type $\tau$
\item $\{\varphi_1, \ldots, \varphi_n\}$ is overloaded function type
\item $\tau\langle\overline{\tau}\rangle$ is type instantiation
\item $\tau.\ell$ is delayed attribute access
\end{itemize}

\paragraph{Field Keys.} Field keys $f \in \mathsf{FieldKey}$ support both positional and nominal access:
\[f ::= \langle i \rangle \mid \langle s \rangle \mid \langle i, s \rangle\]
where $i \in \mathbb{N}$ is a positional index and $s \in \mathbb{S}$ is a field name.

\paragraph{Effects.} Effect annotations $\epsilon \in \mathcal{E}$ are tuples:
\[\epsilon = (\mathsf{new} : \mathbb{B}, \mathsf{method} : \mathbb{B}, \mathsf{update} : \tau^?, \mathsf{alias} : \mathbb{B}, \mathsf{property} : \mathbb{B})\]
where $\tau^?$ denotes an optional type for mutation targets.

\paragraph{Examples.}
\begin{itemize}
\item Function \texttt{def f(x: int, y: str = "d")} has type:
 \[\forall \{\}.\{\langle 0, \text{x} \rangle : \mathsf{int}, \langle 1, \text{y} \rangle : \mathsf{str}\} \xrightarrow{\epsilon_\mathsf{pure}} \mathsf{None}\]
\item Generic \texttt{list[int]} is represented as $\mathsf{list}\langle\mathsf{int}\rangle$
\item Class with both position and name: \texttt{class Point} has record $\{\langle 0, \text{x} \rangle : \mathsf{int}, \langle 1, \text{y} \rangle : \mathsf{int}\}$
\end{itemize}

\subsection{Static Semantics}

\paragraph{Type Environment.} Let $\Gamma$ map names to types and $\Delta$ map type variables to their bounds.

\paragraph{Subtyping.} The subtype relation $\Gamma \vdash \tau_1 <: \tau_2$ is defined by:

\[
\frac{}{\Gamma \vdash \tau <: \tau} \quad \text{(Refl)}
\]

\[
\frac{\Gamma \vdash \tau_1 <: \tau_2 \quad \Gamma \vdash \tau_2 <: \tau_3}{\Gamma \vdash \tau_1 <: \tau_3} \quad \text{(Trans)}
\]

\[
\frac{\forall i. \Gamma \vdash \tau_i <: \tau}{\Gamma \vdash (\tau_1 + \cdots + \tau_n) <: \tau} \quad \text{(Union-Sub)}
\]

\[
\frac{\mathsf{dom}(R_2) \subseteq \mathsf{dom}(R_1) \quad \forall f \in \mathsf{dom}(R_2). \Gamma \vdash R_1(f) <: R_2(f)}{\Gamma \vdash R_1 <: R_2} \quad \text{(Record-Sub)}
\]

For classes:
\[
\frac{\mathsf{protocol} = \mathsf{false} \quad C_2 \in \mathsf{supers}^*(C_1)}{\Gamma \vdash \mathsf{class}(C_1, \ldots) <: \mathsf{class}(C_2, \ldots)} \quad \text{(Nominal)}
\]

\[
\frac{\mathsf{protocol} = \mathsf{true} \quad \Gamma \vdash R_1 <: R_2}{\Gamma \vdash \mathsf{class}(C_1, R_1, \ldots, \mathsf{true}) <: \mathsf{class}(C_2, R_2, \ldots)} \quad \text{(Structural)}
\]

\paragraph{Type Operations.}

\emph{Join} $\tau_1 \sqcup \tau_2$ computes least upper bound:
\[
\begin{aligned}
\tau \sqcup \tau &= \tau \\
\tau_1 \sqcup \tau_2 &= \tau_1 + \tau_2 \quad \text{if unrelated} \\
R_1 \sqcup R_2 &= \{f : \tau \mid f \in \mathsf{dom}(R_1) \cap \mathsf{dom}(R_2), \tau = R_1(f) \sqcup R_2(f)\}
\end{aligned}
\]

\emph{Meet} $\tau_1 \sqcap \tau_2$ computes greatest lower bound:
\[
\begin{aligned}
R_1 \sqcap R_2 &= \{f : \tau \mid f \in \mathsf{dom}(R_1) \cup \mathsf{dom}(R_2), \tau = R_1(f) \sqcap R_2(f)\}
\end{aligned}
\]

\paragraph{Unification.} We define the unification judgment $\Delta \vdash \tau_1 \doteq \tau_2 \leadsto \theta$ where $\theta : \mathcal{V} \to \mathcal{T}$ is a substitution:

\[
\frac{X \notin \mathsf{dom}(\theta) \quad X \notin \mathsf{FV}(\tau)}{\Delta \vdash X \doteq \tau \leadsto \theta[X \mapsto \tau]} \quad \text{(Var-L)}
\]

\[
\frac{\Delta \vdash \overline{\tau_1} \doteq \overline{\tau_2} \leadsto \theta}{\Delta \vdash c\langle\overline{\tau_1}\rangle \doteq c\langle\overline{\tau_2}\rangle \leadsto \theta} \quad \text{(Inst)}
\]

\[
\frac{X \notin \mathsf{dom}(\theta) \quad \Delta \vdash \overline{\tau_1} \doteq \overline{\tau_2} \leadsto \theta'}{\Delta \vdash X\langle\overline{\tau_1}\rangle \doteq c\langle\overline{\tau_2}\rangle \leadsto \theta[X \mapsto c] \cup \theta'} \quad \text{(Inst-Var)}
\]

\[
\frac{\Delta \vdash R_1 \doteq R_2 \leadsto \theta_1 \quad \Delta \vdash \tau_1 \doteq \tau_2 \leadsto \theta_2 \quad \epsilon_1 \sqcup_\mathcal{E} \epsilon_2 \text{ defined}}{\Delta \vdash (R_1 \xrightarrow{\epsilon_1} \tau_1) \doteq (R_2 \xrightarrow{\epsilon_2} \tau_2) \leadsto \theta_1 \cup \theta_2} \quad \text{(Fun)}
\]

For records, unification finds a matching between fields:
\[
\frac{\forall f_i \in \mathsf{common}(R_1, R_2). \Delta \vdash R_1(f_i) \doteq R_2(f_i) \leadsto \theta_i}{\Delta \vdash R_1 \doteq R_2 \leadsto \bigcup_i \theta_i} \quad \text{(Record)}
\]

For variadic variables $X^*$:
\[
\frac{X^* \notin \mathsf{dom}(\theta) \quad X^* \notin \mathsf{FV}(\tau_1, \ldots, \tau_n)}{\Delta \vdash X^* \doteq (\tau_1, \ldots, \tau_n) \leadsto \theta[X^* \mapsto (\tau_1, \ldots, \tau_n)]} \quad \text{(Var-Star)}
\]

\paragraph{Implementation Note.} The unification algorithm produces intermediate matching objects that track bound/unbound parameters separately, enabling sophisticated partial application and overload resolution. The rules above abstract over these implementation details while preserving the essential semantics.

\paragraph{Name Resolution.} Let $\mathcal{M}$ be the global module environment. Resolution $\mathsf{resolve}(n, \Gamma)$ proceeds:

\[
\mathsf{resolve}(n, \Gamma) = \begin{cases}
\Gamma(n) & \text{if } n \in \Gamma \\
\mathsf{resolve}(n, \Gamma_{\mathsf{parent}}) & \text{if parent exists} \\
\mathcal{M}[\mathsf{current\_module}](n) & \text{if module-local} \\
\mathcal{M}[\text{builtins}](n) & \text{if builtin} \\
\mathsf{error} & \text{otherwise}
\end{cases}
\]

\subsection{Dynamic Semantics}

\paragraph{Overload Resolution.} For overload $\{\varphi_1, \ldots, \varphi_n\}$ and arguments $\overline{\tau}$:

\[
\mathsf{resolve}(\{\varphi_1, \ldots, \varphi_n\}, \overline{\tau}) = \bigsqcup\{\mathsf{return}(\varphi_i) \mid \mathsf{unify}(\mathsf{params}(\varphi_i), \overline{\tau}) \text{ succeeds}\}
\]

\paragraph{Attribute Access.} The access operation $\mathsf{access}(\tau, \tau')$ is defined:

\[
\frac{R = \mathsf{members}(\tau) \quad \tau' = s \quad f = \langle s \rangle \text{ or } f = \langle i, s \rangle \quad f \in \mathsf{dom}(R)}{\mathsf{access}(\tau, \tau') = R(f)} \quad \text{(Field)}
\]

\[
\frac{\mathsf{access}(\tau, \text{\_\_getitem\_\_}) = \varphi \quad \tau' \in \mathsf{IndexTypes} \quad \mathsf{call}(\varphi, \tau') = \tau''}{\mathsf{access}(\tau, \tau') = \tau''} \quad \text{(Index)}
\]

where $\mathsf{IndexTypes} = \{\mathsf{int}, \mathsf{str}, \mathsf{slice}\} \cup \{\ell \mid \ell \text{ literal}\}$.

\paragraph{Effect System.} Effects compose via pointwise join $\sqcup_\mathcal{E}$:

\[
\begin{aligned}
\epsilon_1 \sqcup_\mathcal{E} \epsilon_2 = (&\epsilon_1.\mathsf{new} \lor \epsilon_2.\mathsf{new}, \\
&\epsilon_1.\mathsf{method} \lor \epsilon_2.\mathsf{method}, \\
&\epsilon_1.\mathsf{update} \sqcup^? \epsilon_2.\mathsf{update}, \\
&\epsilon_1.\mathsf{alias} \lor \epsilon_2.\mathsf{alias}, \\
&\epsilon_1.\mathsf{property} \lor \epsilon_2.\mathsf{property})
\end{aligned}
\]

where the optional type join $\sqcup^?$ is defined as:
\[
\tau_1^? \sqcup^? \tau_2^? = \begin{cases}
\tau_1 \sqcup \tau_2 & \text{if both } \tau_1, \tau_2 \text{ defined} \\
\tau_i & \text{if only } \tau_i \text{ defined} \\
\bot & \text{if neither defined}
\end{cases}
\]

\paragraph{Integration with Analysis.} The type system provides semantic foundations for:
\begin{itemize}
\item \textbf{Heap Shape}: Record types $R$ determine object layout in pointer analysis
\item \textbf{Effect Tracking}: Effect annotations $\epsilon$ guide mutation and aliasing analysis  
\item \textbf{Dynamic Resolution}: Access operations $\tau.\ell$ resolve attribute lookups during abstract interpretation
\item \textbf{Allocation Sites}: Functions with $\epsilon.\mathsf{new} = \mathsf{true}$ create fresh heap locations
\item \textbf{Row Polymorphism}: Field keys with both positional and nominal components enable precise modeling of Python's parameter passing semantics
\end{itemize}
