\clearpage
\appendix
\section{Analysis Assumptions and Design Trade-offs}
\label{sec:appendix-assumptions}

To ensure a sound and tractable static analysis of a highly dynamic language, our framework targets a well-defined and practical subset of Python programs. This section outlines the scope of our analysis and the key design trade-offs that balance implementation complexity with analytical precision.

\paragraph{Target Program Scope}
Our analysis requires that programs adhere to the following properties, which enable a sound interpretation of control and data flow:
\begin{itemize}
    \item \textbf{No Dynamic Code Evaluation:} Constructs such as \texttt{eval}, \texttt{exec}, and \texttt{getattr} with dynamically computed string arguments are disallowed. \texttt{getattr} with literal strings (e.g., \texttt{getattr(x, "mean")}) is supported, as it's semantically equivalent to a standard attribute access (\texttt{x.mean}). These features prevent the static resolution of the program's control-flow graph.
    \item \textbf{Statically-Resolvable Calls:} Function calls must be resolvable at analysis time using the provided type signatures. The framework does not model complex higher-order control flow where functions are passed as first-class values to unknown call sites.
    \item \textbf{Explicit Generic Instantiations:} To avoid relying on runtime type propagation, generic collections must be explicitly instantiated with their type parameters (e.g., \texttt{list[int]()}), especially when empty.
    \item \textbf{Simplified NumPy Aliasing:} The analysis assumes that NumPy array variables refer to distinct memory objects unless explicitly constructed via view-creating operations. It does not model view-based aliasing where multiple arrays may share the same underlying data buffer.
    \item \textbf{Operator implementation is regular:} operator lookup rules in Python are complex. Usually, however, one can assume that all candidate implementations do generally the same thing. Similar assumptions are required for soundness in type checkers such as MyPy.
    \item \textbf{No reentrance} function called can be overapproximated using effect annotation, and do not depend on the behavior of the callsite except the information explicitly passed through arguments and self-binding.
\end{itemize}

These assumptions are satisfied by a wide class of numerical programs that follow idiomatic NumPy usage: preallocated buffers, explicit data copying, no implicit sharing, and simple iteration over typed arrays. These assumptions enable a sound, precise static analysis tailored to checkpointing in deterministic, structured Python programs.

\paragraph{Precision and Design Trade-offs}
Within this scope, our analysis embodies several conscious design trade-offs. Some choices simplify the abstract domain at the cost of precision, while others introduce complexity to the type system to more faithfully model Python's idioms and avoid hardcoding.
\begin{itemize}
    \item \textbf{Dimensionality-Agnostic Array Types:} The type system abstracts all \texttt{numpy.ndarray} objects as containing floating-point numbers but does not track their dimensionality or shape. This design greatly simplifies the typing of numerical operations but means the analysis cannot distinguish between a vector and a matrix, which in some cases may require additional user hints to ensure type precision.
    \item \textbf{Wildcard for Collection Elements:} To handle collections of arbitrary size and for accesses that are not precisely known, our pointer analysis models all element access (e.g., via subscripting) using a single wildcard field, \texttt{*}. This is efficient and scalable but merges the abstract state of all elements, meaning a write to one index will appear to affect all others.
    \item \textbf{Literal Types for Precision:} We chose to add complexity by incorporating literal types (e.g., \texttt{Literal["mean"]}) into the type system. While this makes the type hierarchy more complex, it enables a fully generic, type-driven model for attribute access. It allows the analysis to resolve expressions like \texttt{x.mean} by treating it as a subscription on \texttt{x}'s type with the literal value, avoiding hardcoded heuristics for method names.
    \item \textbf{Variadic Generics for Generality:} The type system supports variadic generics (e.g., \texttt{*Args}). This required a more complex unification algorithm but was useful to accurately model common Python constructors like \texttt{tuple()} without special-casing them in the analyzer. This design allows the framework to be more extensible and handle a wider range of idiomatic Python code in a principled way.
    \item \textbf{Reliance on Annotations:} The soundness of the analysis is contingent on the correctness and completeness of the provided type and side-effect annotations (e.g., \texttt{new}, \texttt{update}). This is particularly true for external library functions, which are modeled as black boxes whose behavior is determined entirely by these summaries. Verifying these annotations is currently a manual process.
\end{itemize}

