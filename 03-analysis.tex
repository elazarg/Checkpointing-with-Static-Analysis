\section{Analysis}
\label{sec:analysis}

Our static analysis framework combines several interacting domains to determine the minimal set of program variables that must be included in a checkpoint.  
We present the domains in an order that matches their conceptual dependencies, starting from the IR, then moving to liveness, explaining objects, and general domain conventions before introducing the specific domains for dirty tracking, pointers, and types.

\subsection{Intermediate Representation}
Our analyses operate not on Python source or bytecode directly, but on a simplified register-based \emph{three-address code} (TAC).
This IR is designed to make dataflow and heap accesses explicit, removing the operand stack of Python bytecode and any implicit temporaries.
A TAC program consists of a finite set of \emph{labels} $L$, each with an associated basic block of instructions.
Each instruction reads and writes explicit \emph{variables} $v \in \mathcal{V}$, which may be:
\begin{itemize}
    \item \emph{Named locals} (e.g.\ \texttt{x}, \texttt{result}), corresponding to function parameters and Python local variables.
    \item \emph{Stack variables} ($\$0, \$1, \ldots$), introduced during translation to replace Python's stack slots.
\end{itemize}
Control flow is represented explicitly in a \emph{control-flow graph} (CFG) $G = (L, E)$ with directed edges for possible execution paths.
We write $\mathsf{succ}(l)$ and $\mathsf{pred}(l)$ for the successors and predecessors of a label $l$ in $G$.
An \emph{instruction location} is denoted by $(l, i)$ for the $i$-th instruction in block $l$.
\subsection{Liveness Analysis}
Liveness analysis determines, for each program point, the set of TAC variables whose current value may be read along some future control-flow path before being overwritten.  
We compute liveness as a standard backward dataflow analysis on the TAC, with the set of live variables at a location $(l,i)$ denoted $\mathsf{Live}_{l,i} \subseteq \mathcal{V}$.

Related: \cite{pythonsemantics}.

\paragraph{Gen/Kill sets.}
For each instruction, we define:
\begin{itemize}
    \item $\mathsf{gen}$: the set of variables read by the instruction (whose current value is needed now).
    \item $\mathsf{kill}$: the set of variables definitely overwritten by the instruction (whose current value is no longer needed).
\end{itemize}
The backward transfer function is:
\[
\mathsf{Live}_{\mathrm{in}} = (\mathsf{Live}_{\mathrm{out}} \setminus \mathsf{kill}) \cup \mathsf{gen}.
\]
A variable in $\mathsf{kill}$ is said to be \emph{killed} at that instruction.

\paragraph{Variable kinds.}
Variables in $\mathcal{V}$ come in two forms:
\begin{itemize}
    \item \emph{Stack variables} ($\$0, \$1, \ldots$) are compiler-generated temporaries that exist only within a single basic block and cannot be addressed once they are dead.  
          Killing such a variable immediately removes any heap objects reachable from it from the root set.
    \item \emph{Named local variables} (e.g.\ \texttt{x}, \texttt{result}) persist for the duration of a function activation and may, in Python, be captured by closures or declared \texttt{nonlocal}.
          We restrict our target program subset to disallow such captures, so a named local can be considered dead when no further reads occur in the current function body.
\end{itemize}

\paragraph{Role in later domains.}
The live-variable set at each point is used to prune the root set for the pointer graph: when a variable $v$ is not live, all edges reachable only through $v$ are removed.
This prevents dead variables from keeping otherwise unreachable objects alive in the analysis state.

\subsection{Objects and Fields}
We model the heap as a finite set of \emph{abstract objects} $\mathcal{O}$, each representing either:
\begin{itemize}
    \item A distinguished \emph{root object} such as \texttt{LOCALS} or \texttt{GLOBALS}.
    \item A \emph{parameter object} representing a function argument at entry.
    \item An \emph{allocation site object} $o_{l,i}$ created by a specific instruction location $(l, i)$.
\end{itemize}
Objects have \emph{fields} $f \in \mathcal{F}$ representing attributes, dictionary keys, or sequence elements.
For collections, we use a wildcard field $\star$ to conservatively represent all indexed elements.
Accessing \texttt{x.f} refers to field $f$ of the object(s) currently bound to variable $x$.

We will write $P[o][f]$ for the set of target objects stored in field $f$ of object $o$ in the pointer graph (defined later), and $D[o]$ for the set of fields of $o$ that are currently dirty.

\subsection{Map Domains}
Many of our domains are \emph{maps}, i.e.\ finite partial functions $M : K \to V$ with a fixed \emph{default value} $\bot_V$ for missing keys.
We adopt a uniform notation and semantics for their use in abstract interpretation:
\begin{itemize}
    \item \textbf{Transfer:} An assignment to key $k$ replaces $M[k]$ with a new value $v$.
    \item \textbf{Weak update:} If $k$ may alias with other keys, we conservatively set $M[k] \leftarrow M[k] \sqcup v$ instead of overwriting.
    \item \textbf{Join:} $(M_1 \sqcup M_2)[k] = M_1[k] \sqcup M_2[k]$ for all $k$ in the union of their domains.
    \item \textbf{Subsumption:} $M_1 \sqsubseteq M_2$ if $M_1[k] \sqsubseteq M_2[k]$ for all $k$ in the union of their domains.
\end{itemize}
We treat missing keys as implicitly mapped to $\bot_V$ for the purposes of all operations above.
In our analyses, the value type $V$ may itself be another map, a set of objects, or a simple lattice such as booleans, making this pattern reusable across domains.

\subsection{Dirty Domain}
The \emph{dirty} domain maps each object to the set of fields written since the last checkpoint.
An update marks the corresponding field as dirty; deletions are treated as writes.
The domain itself does not decide which dirty objects must be checkpointed—this decision is deferred until pointer reachability and liveness are taken into account.
\subsection{Pointer Domain}
The \emph{pointer domain} models the shape of the heap as a map
\[
P : \mathcal{O} \to (\mathcal{F} \to \mathcal{P}(\mathcal{O}))
\]
where $P[o][f]$ is the set of abstract objects that may be stored in field $f$ of object $o$.
The inner map has default value $\emptyset$ for any field not explicitly present.

\paragraph{Transfer.}
On an instruction that stores a reference into a field---for example \texttt{x.f = y}---we first resolve the set of possible source objects $S$ for $y$ and the set of possible target objects $T$ for $x$.
For each $t \in T$, we update $P[t][f]$ with $S$.
If $t$ is known to be the only object bound to $x$ (no aliasing), we perform a \emph{strong update}:
\[
P[t][f] \leftarrow S.
\]
If aliasing is possible, we perform a \emph{weak update}:
\[
P[t][f] \leftarrow P[t][f] \cup S.
\]
Reads, such as \texttt{y = x.f}, do not change $P$ but are resolved by looking up $\bigcup_{t \in T} P[t][f]$.

\paragraph{Join and Subsumption.}
The join $P_1 \sqcup P_2$ is computed pointwise over objects and fields, taking the union of target sets.
Subsumption $P_1 \sqsubseteq P_2$ holds if $P_1[o][f] \subseteq P_2[o][f]$ for all $o \in \mathcal{O}$ and $f \in \mathcal{F}$.

\paragraph{Reachability.}
Given a set of \emph{root objects} $R \subseteq \mathcal{O}$ (typically \texttt{LOCALS} and \texttt{GLOBALS}), the set of \emph{reachable objects} $\mathsf{Reach}(R, P)$ is the smallest set containing $R$ and closed under:
\[
o \in \mathsf{Reach}(R, P) \wedge o' \in P[o][f] \implies o' \in \mathsf{Reach}(R, P).
\]
We use liveness to prune the roots $R$ before reachability is computed, ensuring that dead variables do not keep their objects alive.

\subsection{Type and Effect Domain}

The type domain classifies abstract objects using a type language designed for Python’s object model but specialised for well-structured, numerically oriented code.
It captures not just nominal classes, but also structural information (via row types and protocols), callable signatures, and heap effects.

\paragraph{Type syntax.}
The main constructors (see Appendix~\ref{appendix:typesystem} for the full grammar) include:
\begin{itemize}
\item \textbf{Classes} $\mathsf{class}(C, R, \overline{\tau}, \overline{\alpha})$: a nominal class $C$ with a row $R$ of fields and optional generic parameters.
\item \textbf{Protocols} $\mathsf{protocol}(R, \overline{\alpha})$: structural interfaces that require the listed fields with given types, independent of nominal class.
\item \textbf{Modules}: rows of exported names and their types.
\item \textbf{Callable types} $\forall \overline{X}.;R\_{\mathrm{params}} \xrightarrow{\epsilon} \tau\_{\mathrm{ret}}$: with parameter and return rows, an effect annotation $\epsilon$, and possible overload sets.
\item \textbf{Unions}, \textbf{generic instantiations}, and \textbf{variadic parameters} for \texttt{\*args}-style functions.
\item \texttt{any} for unknown values.
\end{itemize}

\subsection{Type parametricity}
\label{subsec:type-parametricity}

This section describes the handling of type parameters for \emph{functions} in our type system.
We proceed from the simplest case of non-generic functions to the more complex cases
of partial application, overloaded functions, and functions with explicit type parameters
(including variadic packs).

\paragraph{Non-generic functions.}
A non-generic function case is written as:
\[
(p_0 : \tau_0, \dots, p_{n-1} : \tau_{n-1}) \xrightarrow{\epsilon} \tau_r
\]
where $\epsilon$ is the function's effect annotation (ignored in this section).
A \emph{call} is checked by matching each parameter type $\tau_i$ to the corresponding
argument type. If all parameters are matched, the call returns $\tau_r$.

\paragraph{Overloads.}
An overloaded function is a finite set of cases:
\[
\mathsf{Overloaded}(f_1, \dots, f_m)
\]
Each call is attempted against each arm $f_j$ independently.
Arms whose parameters do not match the actual arguments are discarded.
If more than one arm remains, their return types are joined.
This preserves precision while deferring disambiguation until sufficient arguments are supplied.

\paragraph{Partial application.}
Calls may supply only a subset of the function's parameters.
The type system performs \emph{partial binding}:
\begin{enumerate}
    \item Match each supplied parameter against the corresponding argument type.
    \item Apply the resulting substitution to all parameter and return types.
    \item Remove matched parameter rows, reindexing the remainder.
    \item Return a \emph{residual callable} with the remaining parameters and updated types.
\end{enumerate}
When all parameters are matched, the call immediately yields the substituted return type.

\paragraph{Function-level type parameters.}
A generic function case has the form:
\[
\forall X_1, \dots, X_k .\; (p_0 : \tau_0, \dots, p_{n-1} : \tau_{n-1}) \xrightarrow{\epsilon} \tau_r
\]
Type parameters may be:
\begin{itemize}
    \item Declared explicitly in the function's generic header.
    \item Introduced implicitly when a free type variable appears in a parameter type.
\end{itemize}
Type parameters are scoped to the function case and may shadow outer variables.

On a call, \emph{unification} between $\tau_i$ and the actual argument type may
\emph{solve} some type parameters, yielding a substitution $\theta$.
This substitution is applied to all parameter and return types.
Parameters whose type variables are fully solved and no longer appear in the residual type
are dropped from the quantifier list, except for variadic packs that remain in unmatched parameters.

\paragraph{Variadic packs.}
A variadic pack $X^*$ may appear in the parameter list to represent an arbitrary
sequence of contiguous positional arguments.
When matching:
\begin{enumerate}
    \item All remaining positionals are collected and joined elementwise into a $Star(\dots)$ type.
    \item The pack variable is bound to this $Star(\dots)$ in the substitution.
    \item The variadic parameter remains in the residual callable to allow further arguments to be supplied.
\end{enumerate}
If a pack is the only remaining parameter, the case is considered callable-empty,
allowing the call to complete with zero further arguments.
Pack variables also appear in return types or other parameters and can be \emph{indexed}
via $Access$ expressions, which are resolved once the pack is concretized.

\paragraph{Overloads with generics and partial application.}
When the callee is an overloaded function, partial binding is attempted
for each arm independently.
Residuals from arms that match the same set of supplied parameters are grouped together.
If all residuals in a group are callable-empty, their return types are joined
to produce the call's result; otherwise the result is a residual overload.

\paragraph{Self-binding for methods.}
A method type is a function case whose first parameter is the receiver:
\[
\forall \overline{X}.\; (0 : \tau_{\mathit{self}},\; 1:\tau_1, \dots, k:\tau_k) \xrightarrow{\epsilon} \tau_r
\]
Given a receiver type $\sigma_{\mathit{recv}}$:
\begin{enumerate}
    \item Unify $\tau_{\mathit{self}}$ with $\sigma_{\mathit{recv}}$, yielding a substitution $\theta$.
    \item Apply $\theta$ to the remaining parameters, return type, and type parameter list.
    \item Remove the receiver row and reindex remaining parameters.
    \item Drop solved type parameters that no longer appear in the residual type.
\end{enumerate}
The result is a callable that expects only the explicit arguments after \texttt{self}.


\paragraph{Row types and row variables.}
A \emph{row type} is a finite list of fields, indexed by name and/or position, with associated types.
A \emph{row variable} $\rho$ denotes “all other fields” and allows open records.
Row polymorphism supports \emph{width subtyping}: a type with extra fields is a subtype of one with fewer fields, provided the shared fields match.
Protocols are a special case of this — any object with the required fields is a subtype, regardless of its nominal class.

\paragraph{Joins.}
The join operation merges type information:
\begin{itemize}
\item \textbf{Unions}: $\mathsf{union}$ of alternatives.
\item \textbf{Classes with same name}: join each field type; if fields disagree in presence or variance, widen to \texttt{any} for that field.
\item \textbf{Classes with different names}: join to a union type.
\item \textbf{Effects}: union the set of updates, join result types, and OR allocation/points-to flags.
\end{itemize}
Joins introduce imprecision when element types disagree or when residual rows widen to \texttt{any}.

\paragraph{Effects.}
Effects annotate callable types with heap interactions:
\begin{itemize}
\item \textbf{new} — allocates a fresh object of a given type.
\item \textbf{points-to} — returns a reference to an existing object.
\item \textbf{update} — modifies a specific field or refines its type.
\end{itemize}
Updates are instantiated with actual arguments. For example, from the builtins stub:
\begin{align*}
\texttt{list.append} &: (\texttt{self}:\mathrm{list}[T],\ \texttt{x}:S) \to \texttt{None} \\
\mathrm{with}\ & \mathrm{update}[\texttt{self},\ 0:\mathrm{list}[T \mid S]]
\end{align*}

If \texttt{self} starts as \texttt{list[$\bot$]} and the call is \texttt{append(x)} where x is of type \texttt{int}, the element type is refined to \texttt{list[int]}. Note that this is not a special case for list; any library can use this effect.

\paragraph{Monomorphic \texttt{self} requirement.}
Type-changing \texttt{update} effects are only applied when the receiver object is unaliased and monomorphic.
Without this restriction, a refinement on one alias could unsoundly affect unrelated references.

\paragraph{Overloads and dispatch.}
An $\mathsf{overload}(\varphi\_1,\dots,\varphi\_n)$ is a finite set of callable types differing in parameter rows, return types, or effects.
At a call site:
\begin{enumerate}
\item Type variables are instantiated from argument types.
\item Variadic arguments are expanded if present.
\item Row subtyping is checked.
\end{enumerate}
Matching overloads contribute their return types and effects; multiple matches are joined; no match yields \texttt{any}.

\paragraph{Field and attribute access.}
For $x.f$, the pointer domain supplies the possible receivers; each receiver’s row is inspected:
\begin{itemize}
\item All agree $\Rightarrow$ take that type.
\item Disagree $\Rightarrow$ join types.
\item Missing $\Rightarrow$ join with the residual row’s default (often \texttt{any}) and mark as possibly failing.
\end{itemize}

\paragraph{Precision and limits.}
The type system can:
\begin{enumerate}
\item Identify exactly which heap locations an operation may modify.
\item Exclude immutable structures from checkpoints.
\item Refine container element types across updates.
\end{enumerate}
Precision loss occurs when:
\begin{itemize}
\item Joins on differing field types force unions.
\item Residual rows accumulate many unknowns, widening to \texttt{any}.
\item Computed attributes (\texttt{getattr} with unknown names) require falling back to \texttt{any}.
\end{itemize}
Dynamic Python features like monkey-patching, dynamic module imports, or metaclass trickery are excluded from the target subset.


\subsection{Domain Interactions}
The domains operate in a reduced product with directed information flow:
\begin{itemize}
    \item \emph{Liveness $\rightarrow$ Pointer:} prunes unreachable roots and their edges.
    \item \emph{Type $\rightarrow$ Pointer:} constrains possible targets based on type-level field shape.
    \item \emph{Type $\rightarrow$ Dirty:} immutable objects never become dirty.
    \item \emph{Pointer $\rightarrow$ Type:} dynamic dispatch resolution is narrowed to types of reachable receivers.
\end{itemize}
These interactions improve precision by removing impossible states before they can pollute other domains.

\subsection{Transfer Functions}
For each TAC instruction, the transfer function updates only the affected keys in each domain according to the instruction’s semantics.
Weak updates are applied when aliasing is possible, and joins are used when merging control-flow paths.
The set of checkpoint roots at loop boundaries is obtained by intersecting live roots with reachable dirty objects.

