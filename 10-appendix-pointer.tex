\newcommand{\defeq}{\overset{\text{\tiny def}}{=}}

\begin{figure*}[t]
\centering

\begin{minipage}[t]{0.5\textwidth}
\centering
\textbf{Heap Domain ($H$)}
\begin{gather*}
H \in \mathcal{H} = \tO \to (\tK \to \powerset(\tO)) | \bot_\mathcal{H} \\[0.3em]
P_1 \sqsubseteq P_2 \iff \forall o.\; P_1[o] \subseteq P_2[o] \\[0.3em]
(P_1 \sqcup P_2)[o] = P_1[o] \cup P_2[o] \\
\end{gather*}
\end{minipage}%
\hfill
\begin{minipage}[t]{0.5\textwidth}
\centering
\textbf{Stack Domain ($S$)}
\begin{gather*}
S \in \mathcal{S} = \tN \to \powerset(\tO) | \bot_\mathcal{S} \\[0.3em]
S_1 \sqsubseteq S_2 \iff \forall i \in \tN.\; S_1[i] \subseteq S_2[i] \\[0.3em]
(S_1 \sqcup S_2)[i] = S_1[i] \cup S_2[i]
\end{gather*}
\end{minipage}

\vspace{1.5em}

\begin{minipage}[t]{0.5\textwidth}
\centering
\textbf{Type Domain ($T$)}
\begin{gather*}
T \in \mathcal{T} = \tO \to \mathit{TypeExpr} | \bot_\mathcal{T} \\[0.3em]
T_1 \sqsubseteq T_2 \iff \forall o.\; T_1[o] \leq: T_2[o] \\[0.3em]
(T_1 \sqcup T_2)[o] = T_1[o] \sqcup T_2[o]
\end{gather*}
\end{minipage}%
\hfill
\begin{minipage}[t]{0.5\textwidth}
\centering
\textbf{Dirty Domain ($D$)}
\begin{gather*}
D \in \mathcal{D} = \tO \to \powerset(\tK) | \bot_\mathcal{D} \\[0.3em]
D_1 \sqsubseteq D_2 \iff \forall o.\; D_1[o] \subseteq D_2[o] \\[0.3em]
(D_1 \sqcup D_2)[o] = D_1[o] \cup D_2[o]
\end{gather*}
\end{minipage}

\vspace{1.5em}

\noindent\textbf{Combined domain ($\Sigma$)}
\begin{align*}
\Sigma &= (H,S,T,D) \in \mathcal{H} \times \mathcal{S} \times \mathcal{T} \times \mathcal{D} | \bot_\Sigma\\[0.3em]
\Sigma_1 \sqsubseteq \Sigma_2 &\iff P_1 \sqsubseteq P_2 \land S_1 \sqsubseteq S_2 \land T_1 \sqsubseteq T_2 \land D_1 \sqsubseteq D_2 \\[0.3em]
\Sigma_1 \sqcup \Sigma_2 &= (P_1 \sqcup P_2,\; S_1 \sqcup S_2,\; T_1 \sqcup T_2,\; D_1 \sqcup D_2)
\end{align*}

\caption{Abstract domains for heap analysis combining pointer tracking, type information, and mutation tracking.}
\label{fig:abstract-domains}
\end{figure*}

\newpage
\section{Pointer Analysis}
\label{sec:appendix-pointer}

\subsection{Scope and Goal}

The pointer analysis abstracts heap shape, aliasing, and mutations over the \spytecode IR
(\Cref{sec:appendix-tac-ir}) to support checkpointing. It is
flow-sensitive and intraprocedural along exception-free paths.
Instruction–level effects (allocation at \textsf{Bind}, \textsf{ConstructTuple},
\textsf{ConstructDict}, \textsf{Call}, and iteration in \textsf{Unpack}) are interpreted here.
Callable \emph{invocation} effects come from the type system
(\Cref{sec:appendix-typesystem}) as an effect triple
\(\epsilon^\#=(\tnew,\tupdate,\tptstoargs)\) and are translated to
heap updates by the transfer functions below.

\subsection{Notation and State}

We write basic blocks \(l\in L\) and instruction positions \(i\) within block \(l\), giving a program point \(l\).
Variables are split into named locals \(\mathcal{V}_{\mathrm{named}}\) and stack stack indices $\tN$.
Let \(\mathsf{Live}_{l,i}\subseteq \mathcal{V}_{\mathrm{named}}\) be liveness at \(l\).

\paragraph{Abstract objects and fields.}
Abstract objects \(\tO\) include
\begin{enumerate}
    \item \loc{l} --- allocation site.
    \item \tParam p --- external input.
    \item \tImm $\tau$ --- immutable of type $\tau$.
    \item scopes ---\tLOCALS, \tGLOBALS.
\end{enumerate}

Fields \(\tK\) are keys from the IR:
\(\mathsf{name}(s)\), \(\mathsf{index}(n)\), \(\mathsf{both}(n,s)\), and \(\star\) (wildcard).

\paragraph{Abstract domains (cf.\ Fig.~\ref{fig:abstract-domains}).}

\paragraph{Conventions.}
We store named local variables as fields of roots:
\(H[\mathtt{LOCALS}][\mathsf{name}(x)]\) for \(x\in\mathcal{V}_{\mathrm{named}}\),
and \(S[\$k]\) for \(\$k\in\tN\).
We write \(O_x\) for a set of objects and \(\biguplus\) for union over a family.

\subsection{Auxiliary Updates}

Weak/strong update on points-to:
\[
\mathsf{Upd}(H,O_{\mathrm{tgt}},f,O_{\mathrm{new}})=
\begin{cases}
\text{for the unique }o\in O_{\mathrm{tgt}}:~H[o][f]\leftarrow O_{\mathrm{new}} & \text{if }|O_{\mathrm{tgt}}|=1\\
\text{for all }o\in O_{\mathrm{tgt}}:~H[o][f]\leftarrow H[o][f]\cup O_{\mathrm{new}} & \text{otherwise}
\end{cases}
\]
Dirty update: \(\mathsf{UpdD}(D,O_{\mathrm{tgt}},f)\) adds \(f\) to \(D[o]\) for each \(o\in O_{\mathrm{tgt}}\).

\smallskip
We use two auxiliary resolvers:
\[
\mathsf{RetObjects}(l,i,\epsilon^\#,\overline{O_a}) =
\begin{cases}
\{\loc{l}\} & \text{if }\epsilon^\#.\tnew=\mathsf{true}\\
\Big(\biguplus_k O_{a_k}\Big)\ \cup\ \{\mathsf{Ret}^\uparrow\} & \text{otherwise}
\end{cases}
\]
(where \(\mathsf{Ret}^\uparrow\) is a distinguished summary return object),
and a target resolver for effect updates that follows argument/receiver links on bound callables:
\[
\mathsf{Targets}(H, O_f, \text{slot}) \subseteq \tO
\quad\text{(slot is receiver or an argument index)}.
\]

\subsection{Transfer Functions for \spytecode}
\label{subsec:transfers}

The abstract state is a four-tuple \(\Sigma = (H,S,T,D)\) where \(H\) is the heap
points-to map, \(S\) is the
stack map, \(T\) the object-type map, \(D\) the dirty-field map.
We write \(S[d \mapsto O]\) for functional update, and
\(\biguplus\mathcal{X}\) for set union over a family \(\mathcal{X}\).
Heap updates use
\(\mathsf{Upd}(H,\ O_{\mathrm{tgt}},\ f,\ O_{\mathrm{new}})\)
(weak/strong update depending on \(|O_{\mathrm{tgt}}|\)),
and dirty marks use
\(\mathsf{UpdD}(D,\ O_{\mathrm{tgt}},\ f)\).
Let \(\loc{l}\) denote the allocation site at instruction index \(l\).
Pure operations never mutate \(H\); the stack map \(S\) changes only for
instructions that assign to a stack.
\subsection{Transfer Functions for \spytecode}
\label{subsec:transfers}

The abstract state is a four-tuple \(\Sigma=(H,S,T,D)\) where \(H\) is the heap
points-to map, \(T\) the object-type map, \(D\) the dirty-field map, and \(S\)
the stack map for stack variables.  Pure operations never mutate \(H\).  Only
instructions that assign to a destination stack variable can mutate \(S\).

\begin{figure}[t]
\centering
\resizebox{\linewidth}{!}{$
\setlength{\jot}{0.35em}% tighten vertical gaps between aligned rows
\begin{aligned}
&\textbf{(A) Data movement}\\
&Trans^\#\big((H,S,T,D),\mathsf{Mov}(\$d,\$s)\big)
  = (H,\ S\llbracket d \mapsto S[s]\rrbracket,\ T,\ D)
\\[0.6em]
&\textbf{(B) Constant lookup}\\
&Trans^\#\big((H,S,T,D),\tLoadConst(\$d,c)\big)
  = (H,\ S\llbracket d \mapsto \{\,\mathsf{Imm}(type(c))\,\}\rrbracket,\ T,\ D)
\\
&Trans^\#\big((H,S,T,D),\tLookupDunder(\$d,op,\$a)\big)
  =(H,S,T,D\llbracket d \mapsto \mathsf{lookup\_dunder}(T, op, S[\$a])\rrbracket)
\\
&Trans^\#\big((H,S,T,D),\tResolveOverload(\$d,\$f,\$a,\$k)\big)
  =(H,S,T,D\llbracket d \mapsto \mathsf{resolve\_overload}(T, \,f, \,S[\$a], \,S[\$k])\rrbracket)
\\[0.6em]
&\textbf{(C) Attribute read and named variable access (pure w.r.t.\ heap)}\\
&Trans^\#\big((H,S,T,D),\tGetAttr(\$d,\$o,f)\big)
  = \big(H,\ S\llbracket d \mapsto \mathsf{Attr}(H,S[o],f)\rrbracket,\ T,\ D\big)
\\
&Trans^\#\big((H,S,T,D),\tLoadLocal(\$d,x)\big)
  = (H,\ S\llbracket d \mapsto H[\tLOCALS][name(x)]\rrbracket,\ T,\ D)
\\
&Trans^\#\big((H,S,T,D),\tLoadGlobal(\$d,x)\big)
  = (H,\ S\llbracket d \mapsto H[\tGLOBALS][name(x)]\rrbracket,\ T,\ D)
\\[0.6em]
&\textbf{(D) Allocation schemata (allocate \& wire, then write stack)}\\
&Trans^\#\big((H,S,T,D),\tConstructTuple(\$d,\overline{\$e}),\ q\big)
  = \big(H',\ S\llbracket d \mapsto \{\,\loc{q}\,\}\rrbracket,\ T',\ D\big)
\\[-0.1em]
&\hspace{4.45cm}\text{where}\quad
  (H',T') \;\defeq\; \mathsf{AllocTuple}(H,T,\loc{q},\ \langle S[\overline{\$e}]\rangle)
\\[0.25em]
&Trans^\#\big((H,S,T,D),\tConstructDict(\$d,\overline{(\$k,\$v)}),\ q\big)
  = \big(H'',S\llbracket d \mapsto \{\,\loc{q}\,\}\rrbracket,\ T'',\ D\ \big)
\\[-0.1em]
&\hspace{4.45cm}\text{where}\quad
  (H'',T'') \;\defeq\; \mathsf{AllocDict}(H,\langle \overline{(S[\$k],S[\$v])}\rangle, T,\loc{q})
\\[0.25em]
&Trans^\#\big((H,S,T,D),\tBind(\$d,\$f,\$a,\$k),\ q\big)
  = \big(H''',\ S\llbracket d \mapsto \{\,\loc{q}\,\}\rrbracket,\ T''',\ D\big)
\\[-0.1em]
&\hspace{4.45cm}\text{where}\quad
  (H''',T''') \;\defeq\; \mathsf{AllocBind}(H,S[k],T,\loc{q},\ S[a])
\\[0.6em]
&\textbf{(E) Heap updates via fields and calls}\\
&Trans^\#\big((H,S,T,D),\tSetLocal(x,\$s)\big)
  = \big(Upd(H,\{\tLOCALS\},name(x),S[s]),\ S,\ T,\ D\big)
\\
&Trans^\#\big((H,S,T,D),\tUnpack(\overline{\$d},\$s)\big)
  = \big(H,\ S',\ T,\ D\big)
\qquad\text{where}\quad
S'[d_j] \;\defeq\; Ind(H,S[s],j)
\\
&Trans^\#\big((H,S,T,D),\tSetAttr(\$o,f,\$v)\big)
  = \big(Upd(H,S[o],name(f),S[v]),\ S,\ T,\ UpdD(D,S[o],name(f))\big)
\\
&Trans^\#\big((H,S,T,D),\tCall(\$d,\$f),\ q\big)
  = \big(H^\ast,\ S\llbracket d \mapsto O_{\mathrm{res}}\rrbracket,\ T^\ast,\ D^\ast\big)
\\[-0.3em]
&\hspace{2.45cm}\text{where}\quad
  \begin{aligned}[t]
  (H^\ast,T^\ast,D^\ast,O_{\mathrm{res}})\ &\defeq\
  \mathsf{Invoke}\big(H,T,D;\ \mathsf{Effect}(\$f),\\
  &\quad S[f],\ \mathsf{Args}(H,S,S[f]),\ q\big)
  \end{aligned}
\\[0.6em]
&\textbf{(F) Control (no state change)}\\
&Trans^\#\big((H,S,T,D),\tAssumeEq(\$b,\$c)\big)=(H,S,T,D),
\quad
Trans^\#\big((H,S,T,D),\tExit\big)=(H,S,T,D)
\end{aligned}
$}
\caption{Abstract transformer \(Trans^\#\).
Only destination-producing instructions write to the stack map \(S\).
Heap \(H\) is mutated at allocation sites (\tConstructTuple, \tConstructDict, \tBind), explicit writes (\tSetLocal, \tSetAttr), and via \tCall according to \(\epsilon^\#\).
\tLoadConst, \tLookupDunder and \tResolveOverload are pure.}
\label{fig:tac-transfers}
\end{figure}

\begin{figure}[t]
\centering
\resizebox{\linewidth}{!}{$
\setlength{\arraycolsep}{0.35em}% tighter math columns
\begin{aligned}
\mathsf{Attr}(H,O,f) &\defeq \biguplus_{o\in O} H[o][name(f)] \\[0.2em]
\mathsf{Ind}(H,O,j) &\defeq \Big(\biguplus_{o\in O} H[o][index(j)]\Big)\ \cup\ \Big(\biguplus_{o\in O} H[o][\star]\Big) \\[0.2em]
\mathsf{AllocTuple}(H,T,t,\langle O_0,\ldots,O_{n-1}\rangle)
  &\defeq \Big(H[t][index(j)] \supseteq O_j,\ H[t][\star]\supseteq O_j\ \forall j;\ T[t]\ \sqcup=\ \mathsf{TupleType}(\langle O_j\rangle)\Big) \\[0.2em]
\mathsf{AllocDict}(H,T,d,\langle (K_m,V_m)\rangle_{m=1}^M)
  &\defeq \Big(
    \begin{aligned}[t]
      &H[d][name(s)] \supseteq V_m &&\text{if } K_m=\{\,\mathsf{Imm}(\texttt{str}(s))\,\};\\
      &H[d][\star]\supseteq V_m   &&\text{otherwise};\\
      &T[d]\ \sqcup=\ \mathsf{DictType}(\langle K_m,V_m\rangle)
    \end{aligned}
  \Big) \\[0.2em]
\mathsf{AllocBind}(H,T,b,O_a,O_k)
  &\defeq \Big(H[b][\texttt{"args"}]\supseteq O_a,\ H[b][\texttt{"kwargs"}]\supseteq O_k;\ T[b]\ \sqcup=\ \mathsf{ResidualCallableType}(b)\Big) \\[0.2em]
\mathsf{Args}(H,S,O_f)
  &\defeq \text{actual argument objects resolved from }S\text{ and binding edges hanging off }O_f \\[0.2em]
\mathsf{Effect}(\$f)
  &\defeq \text{the effect summary }\epsilon^\#\text{ retrieved from the type component at }\$f \\[0.2em]
\mathsf{Invoke}(H,T,D;\ \epsilon^\#,O_f,O_a,q)
  &\defeq \text{apply }\epsilon^\#\text{: allocate results (fresh site } \loc{q}\text{ if needed),}\\
  &\qquad\text{propagate aliasing (}\star\text{ edges) when }\epsilon^\#.\tptstoargs,\ \text{and perform field updates}\\
  &\qquad\text{on targets discovered by }Targets(H,O_f,\cdot)\text{ using } \mathsf{Upd} \text{ and } \mathsf{UpdD};\ \text{also update }T.
\end{aligned}
$}
\caption{Helper definitions for abstract transformer.}
\label{fig:tac-transfers-helpers}
\end{figure}

\paragraph{Notes.}
(i) \emph{Lookup purity.} Results of \tLookupDunder/\tResolveOverload are static; we leave $S$ unchanged and rely on the type component to constrain subsequent \tBind/\tCall.
(ii) \emph{Binding vs.\ calling.} \tBind allocates a bound-callable object \(b\) with explicit \texttt{args}/\texttt{kwargs} edges; \tCall interprets \(\epsilon^\#\) to allocate results, propagate aliasing, and perform effect-guided updates on receivers/arguments discovered via \(Targets\).
(iii) \emph{Subscriptions.} Indexing sugar lowers to \(\_\_getitem\_\_\)/\(\_\_setitem\_\_\) and thus uses the same \tBind/\tCall machinery.

\subsection{Reachability, Pruning, and Dirty Roots}

Roots at \(l\) are
\(R_{l,i}=\{\,H[\mathtt{LOCALS}][\mathsf{name}(x)] \mid x\in \mathsf{Live}_{l,i}\,\}\).
Reachability is the least fixed point
\[
\mathsf{Reach}(R,H)=\mu X.\ R\ \cup\ \{\,o' \mid \exists\,o\in X,\, f\in\tK.\ o'\in H[o][f] \,\}.
\]
Abstract GC restricts to \(V=\mathsf{Reach}(R_{l,i},H)\):
\[
\mathsf{GC}(\Sigma,R_{l,i})=(H|_V,\ T|_V,\ D|_V).
\]
Dirty roots:
\[
\mathsf{DirtyRoots}(\Sigma,R_{l,i})=\big\{\, x\in \mathsf{Live}_{l,i}\ \big|\ \exists o\in \mathsf{Reach}(\{H[\mathtt{LOCALS}][\mathsf{name}(x)]\},H).\ D[o]\neq \emptyset \,\big\}.
\]

\subsection{Soundness}

Let \(\gamma\) map abstract states to sets of concrete heaps. Soundness requires:
\[
\forall \Sigma^\#=(H,S,T,D),\ \forall H\in \gamma(\Sigma^\#),\ \forall o,f.\quad H(o).f \in H[o][f].
\]
Whenever the concrete semantics performs a write \(o.f\leftarrow v\), the abstract transformer must
mark \(f\in D[o]\). Monotonicity of transformers and the product lattice yields a standard
worklist fixpoint and a post-fixpoint invariant. Local simulation lemmas follow the \spytecode
operational semantics (lookups are pure; \(\mathsf{Bind}/\mathsf{Construct*}/\mathsf{Call}/\mathsf{Unpack}\) introduce only the abstract updates shown above).
