
% kmeans-running-example-standalone.tex
\clearpage
\section{Running Example (Trace-Guided): \textsc{K-Means}}

\paragraph{Program sketch.}
Each iteration rebuilds \texttt{clusters}, assigns points to the nearest centroid, computes \texttt{new\_centroids}, checks convergence, and (if not converged) overwrites \texttt{centroids}.
Only \texttt{centroids} must persist across iterations; all other temporaries are recomputed.

\paragraph{Notation.}
We use a simplified snapshot notation at selected program points $(\ell)$:
\begin{itemize}
\item $\Live\_\ell$ --- live named locals.
\item $\Points\_\ell[o]\name{f}$ --- pointer edges from abstract object $o$ via field $f$.
\item $\Dirty\_\ell[o]$ --- dirty fields of object $o$ since the last checkpoint.
\item Objects: $\loc n$ is an allocation site; distinguished roots are \LOCALS{} and \GLOBALS{}.
\end{itemize}
Fields use one key space: $\name{s}$ (attributes), $\idx{i}$ (indices), and wildcard $\wild$ for container elements.
We abbreviate common sites from the trace: $\centroidsobj \triangleq \loc 10$ (initial \texttt{centroids} array), $\clustersobj \triangleq \loc 179$ (outer list-of-lists), $\innerlist \triangleq \loc 212$ (inner list), $\deltaarr \triangleq \loc 304$ (per-sample difference), $\normarr \triangleq \loc 307$ (vector of distances), and $\ypredobj \triangleq \loc 616$ (final labels buffer).

\section*{Waypoints in the Trace}
We excerpt six points; identifiers refer to the original trace file.

\begin{enumerate}
\item \textbf{Init \& loop entry}. Chooses $k$ row indices, slices \texttt{X} to \texttt{centroids} ($\centroidsobj$), allocates empty outer list \texttt{clusters} ($\clustersobj$), builds \texttt{range(max\_iterations)}, and enters the loop.
\item \textbf{Rebuild \texttt{clusters}}. Allocates $k$ inner lists and installs them under $\clustersobj[\wild]\mapsto \innerlist$.
\item \textbf{Per-sample assignment}. Forms $v = X[\texttt{s\_i}] - \texttt{centroids}$ ($\deltaarr$), computes $\|v\|$ across axes ($\normarr$), takes $\argmin$ to get \texttt{r}, and appends \texttt{s\_i} into \texttt{clusters[r]}.
\item \textbf{Centroid recompute}. Builds \texttt{new\_centroids} by mapping \texttt{cluster} $\mapsto$ \texttt{X[cluster].mean(0)} and \texttt{np.array} on the results; checks \texttt{np.allclose(centroids, new\_centroids)}.
\item \textbf{Centroid update}. If not converged, executes \texttt{centroids = new\_centroids}.
\item \textbf{Finalization}. Allocates \texttt{y\_pred} ($\ypredobj$), iterates clusters, fills labels, and returns.
\end{enumerate}

\section*{Analysis Snapshots}
Below, we show $(\Live_\ell, \Points_\ell, \Dirty_\ell)$ at four points. Substitutions on stack temporaries are elided; only named locals appear in $\Live_\ell$. The pointer state is pruned by liveness before reachability.

\paragraph{(S1) Loop head after iteration $i{-}1$.}
\begin{align*}
\Live &= \{\texttt{i},\ \texttt{centroids}\}\\
\Points[\LOCALS]\name{centroids} &= \{\centroidsobj\}\\[-0.25em]
\Points[\LOCALS]\name{i} &= \{\Imm{\texttt{int}}\}\\
\Dirty[\centroidsobj] &\subseteq \{\wild\}\qquad\text{(carried from prior update if any)}\\
\Dirty[\clustersobj] &= \emptyset\qquad\text{(rebuilt each iteration)}
\end{align*}

\paragraph{(S2) After rebuilding \texttt{clusters}.}
\begin{align*}
\Live &= \{\texttt{i},\ \texttt{centroids},\ \texttt{clusters}\}\\
\Points[\LOCALS]\name{clusters} &= \{\clustersobj\}\\
\Points[\clustersobj]\wild &\supseteq \{\innerlist\}\\
\Dirty[\clustersobj] &\ni \wild\qquad\text{(placements of inner lists)}
\end{align*}

\paragraph{(S3) Inside inner loop, after \texttt{r = argmin(...)} and \texttt{clusters[r].append(s\_i)}.}
\begin{align*}
\Live &= \{\texttt{i},\ \texttt{centroids},\ \texttt{clusters},\ \texttt{s\_i}\}\\
\Points[\LOCALS]\name{centroids} &= \{\centroidsobj\},\quad
\Points[\LOCALS]\name{clusters} = \{\clustersobj\},\quad
\Points[\clustersobj]\wild \supseteq \{\innerlist\}\\
\Dirty[\clustersobj] &\ni \wild\qquad\text{(mutating inner lists via \texttt{append})}\\
\Types(\centroidsobj) &\le : \texttt{numpy.ndarray},\quad \Types(\normarr)\le:\texttt{numpy.ndarray}
\end{align*}

\paragraph{(S4) After \texttt{new\_centroids} and \texttt{allclose} decision; then update.}
\begin{align*}
\Live &= \{\texttt{i},\ \texttt{centroids},\ \texttt{clusters},\ \texttt{new\_centroids}\}\\
\Points[\LOCALS]\name{new\_centroids} &\ni \loc nc \quad\text{(fresh array from \texttt{np.array})}\\
\Dirty[\clustersobj] &\ni \wild\quad\text{(writes confined to rebuilt \texttt{clusters})}\\[0.25em]
\text{If not converged: }&\ \ \Live=\{\texttt{i},\ \texttt{centroids}\}\ \text{and}\
\Points[\LOCALS]\name{centroids} \leftarrow \Points[\LOCALS]\name{new\_centroids}.
\end{align*}

\section*{Checkpoint Decision}
At the loop head, the minimal persistent root set is the live \emph{and} dirty state reachable from
\{\LOCALS$.\name{centroids}$, \LOCALS$.\name{i}$\}.
Because \texttt{clusters} is reconstructed each iteration and its mutations do not survive across iterations, it is excluded.
Consequently, the checkpoint consists of the scalar loop index and the \texttt{centroids} array.
Instrumentation places \verb|commit(centroids)| at the loop end and restores \verb|i| via the iteration driver.

\section*{Type/Effect Highlights}
We rely on a small effect library:
\begin{itemize}
\item \texttt{np.array}: $E.\tnew=\mathsf{true}$, result does not alias arguments.
\item \texttt{list.append}: $E.\tupdate=\{\wild\}$ marks the wildcard field of the receiver’s element row.
\item \texttt{np.mean}: pure; \texttt{np.allclose}: pure.
\end{itemize}
These effects guide $\Points$ and $\Dirty$ updates and avoid conservative retention of temporary arrays.

\section*{Outcome}
At the loop head, the combined state $\Sigma=(\Points,\Types,\Dirty)$ yields a minimal persistent set: $\{\texttt{centroids}\}$ (plus loop index) with everything else dead or recomputable.
